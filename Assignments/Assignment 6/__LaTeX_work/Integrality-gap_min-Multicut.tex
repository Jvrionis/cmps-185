\documentclass[11pt]{article}
\usepackage{fullpage,xcolor}
\usepackage{upquote}



\begin{document}

\title{\textbf{Corrections To Approximation Algorithms Paper}}\\
\author{\textit{James M. Vrionis}
        \date{March $9^{th}$ 2018}
        %\texttt{}
       }

\maketitle
\hline


%------------------------------------------------------------------------
\section{Paper Rules}
There are \textit{Three Basic Rules} to keep in mind when writing
a paper.  
\begin{enumerate}
\item Do {\color{red}not} start a sentence with a symbol.
\item Avoid {\color{red}run together} mathematical expressions.
\item Watch out for {\color{red}misplaced symbols}.
\end{enumerate}
%------------------------------------------------------------------------

%------------------------------------------------------------------------
\section{Equation in the Title}


% \begin{center}
% Old
% \end{center}

One of the most common of our violated rules can be found in the 
title of the paper. It is a violation to start a paper with an 
equation.  The title of this paper violates rules 1 and 2, which
can be seen below in {\color{red}red} while a corrected version 
will follow in {\color{blue}blue}.\\ 

\begin{center}
{\color{red}$O$ $\left( \sqrt{\log(n)}\right)$} \textbf{Approximation Algorithms 
for Min UnCut, Min 2CNF Deletion, and Directed Cut Problems}
\end{center}

% A violation from here three basic rules often lecture 10 
% in the title to this article.  



% The title of this paper violates rules 1 and 2. To fix we must add 
% {\color{blue}We give} before the term \\ 
% $O \! \left( \sqrt{\log(n)}\right)$.\\

% \begin{center}
%     New
% \end{center}

\begin{center}
{\color{blue}We give} $O \! \left( \sqrt{\log(n)}\right)$ \textbf{Approximation 
Algorithms for Min UnCut, Min 2CNF Deletion, and Directed Cut Problems}
\end{center}

Going one step further.

\begin{center}
\textbf{Improvement on Approximation Algorithms for Min UnCut, Min 2CNF Deletion, 
and Directed Cut Problems}
\end{center}
%------------------------------------------------------------------------

%------------------------------------------------------------------------
\section{Abstract}
\paragraph{}
 A \textbf{Abstract} is a \emph{self-contained} summary of the 
article used to grab the attention and interest of the reader.
Typical things to avoid are Formulas, References, unsubstantiated 
statements and claims, and in some journals use of  
{\ttfamily"}We{\ttfamily"}.  Violation will be in {\color{red}red} font
while a corrected version will follow in {\color{blue}blue}.\\

\paragraph      
{\color{red}We} give {\color{red}$O \! (\sqrt{\log(n)})$}-approximation 
algorithms for the Min UnCut, Min 2CNF Deletion, Directed Balanced Separator, 
and Directed Sparsest Cut problems. The previously best-known algorithms 
give an {\color{red}$O \! (\log(n))$}-approximation for Min UnCut 
{\color{red}[9]}, {\color{red}Directed Balanced Separator [17]}, 
{\color{red}Directed Sparsest Cut [17]}, and an 
{\color{red}$O \! (\log(n) \log(n) \log(n) )$}-approximation for Min 2CNF 
Deletion {\color{red}[14]}.{\color{red}We} also show that the integrality gap of an 
SDP relaxation. \\


\paragraph{} 
    {\color{blue}
    An improvement on previously best-known approximation algorithms for several 
    combinatorial problems, namely: Min UnCut, Directed Balanced Separator, 
    Directed Sparsest Cut and Min 2CNF Deletion. New methods obtained using 
    semidefinite relaxation on directed graphs with a simpler algorithm that 
    deals with Min UnCut.} \\
%------------------------------------------------------------------------

%------------------------------------------------------------------------
\section{Definition 1}

\paragraph{}
    Definition 1 (Min {\color{red}CSP(F)}). Consider boolean 
    variables {\color{red}$b_1,...,b_n$} and a set of constraints 
    {\color{red}$C$ from $F$}. The goal is to find an assignment 
    that minimizes the number of unsatisfied constraints. \\

    Corrections to the above definition.

    \begin{itemize}
        \item A Minimum Constraint Satisfaction Problem or Min CSP may have just been 
    defined it is somewhat confusing in the definition. 
        \item A set of constraints C from a set F.
        \item Ldots used for boolean variable range.
    \end{itemize}


    Definition 1 (Min CSP($F$)) Consider {\color{blue}a set $C$ of constraints from a 
    set $F$} with boolean variable {\color{blue}$b_1, \ldots, b_n$}. 
    {\color{blue}We want to show} an assignment that minimizes the number of 
    unsatisfied constraints.

    \begin{itemize}
        \item A Minimum Constraint Satisfaction Problem or Min CSP may have just been 
    defined it is somewhat confusing in the definition. 
        \item A set of constraints C from a set F.
        \item Ldots used for boolean variable range.
    \end{itemize}


%------------------------------------------------------------------------

%------------------------------------------------------------------------
\section{Three More Violations}

%------------------------------------------------------------------------
\subsection{How Not to State a Theorem 3.1}

Theorem 3.1. There is a randomized polynomial-time algorithm for finding an 
$O( \sqrt{\log (n)})$ approximation for the Min 2CNF Deletion problem.

\begin{itemize}
    \item The statement of this theorem is {\color{red}not} self-contained.
    \item Statement is vague and unclear.
    \item Min 2CNF Deletion problem is defined earlier. 
\end{itemize}
%------------------------------------------------------------------------

%------------------------------------------------------------------------
\subsection{How Not to State a Theorem 2.1}
Theorem 2.1. There is a randomized polynomial-time algorithm for finding an 
$O( \sqrt{\log (n)})$ approximation for the Min UnCUT problem.

\begin{itemize}
    \item The statement of this theorem is {\color{red}not} self-contained.
    \item Statement is vague and unclear.
    \item Min UnCUT Deletion problem is defined earlier. 
\end{itemize}
%------------------------------------------------------------------------

%------------------------------------------------------------------------

\subsection{{\ttfamily"}Any{\ttfamily"} Violations}

In {\color{red}red} I have identified $3$ of the $4$ 
{\ttfamily"}Any{\ttfamily"} violations. \\

The previously best-known approximation ratio for Min UnCut 
is $O(\log n)$ [9], and the best previously known
approximation for Min 2CNF Deletion is 
$O(\log n \log \log n)$[14]. Both problems are known to be 
Max SNP-hard [18]. The best-known lower bound for Min 
2CNF Deletion is $8\sqrt{5} - 15$ $\approx$ $2.88854$ [7].
Moreover, if the Unique Games Conjecture holds true, then 
Min 2CNF Deletion cannot be approximated within {\color{red}any} 
constant factor [13]. \\

LEMMA 5.1. There is a polynomial-time algorithm for the following 
task. Given {\color{red}any} feasible SDP solution with $\beta$ $=$ 
$\sum_{(i,j)\in E }$ $d(i,j)$, and a vertex $k$ such that the ball of 
squared-radius $1/(8n^{2}$) around $v_{k}$ contains at least $n/2$ vectors
(other than $v_{\not}$), the algorithm finds a cut $(S,\bar{S})$ with
directed expansion at most $O(\beta n)$. \\


THEOREM 6.2 Given $r$, $d=4k$, $k \geq r \geq 2$, there exist a positive 
constant $\epsilon$ $= \epsilon(r)$ so that in {\color{red}any} set of
more than $(2 - \epsilon(r))^{d}$ $(\pm 1)$-vectors there are $r$ 
pairwise orthogonal vectors.\\


Corrected {\ttfamily"}Any{\ttfamily"} violations will be shown below
in {\color{blue}blue}. \\

The previously best-known approximation ratio for Min 
UnCut is $O(\log n)$ [9], and the best previously known
approximation for Min 2CNF Deletion is 
$O(\log n \log \log n)$[14]. Both problems are known to be 
Max SNP-hard [18]. The best known lower bound for Min 
2CNF Deletion is $8\sqrt{5} - 15$ $\approx$ $2.88854$ [7].
Moreover, if the Unique Games Conjecture holds true, then 
Min 2CNF Deletion cannot be approximated within 
{\color{blue}every} constant factor [13]. \\


LEMMA 5.1. There is a polynomial-time algorithm for the following 
task. Given {\color{blue}an arbitrarily} feasible SDP solution 
with $\beta$ $=$ $\sum_{(i,j)\in E }$ $d(i,j)$, and a vertex $k$ 
such that the ball of squared-radius $1/(8n^{2}$) around $v_{k}$ 
contains at least $n/2$ vectors (other than $v_{\not}$), the 
algorithm finds a cut $(S,\bar{S})$ with directed expansion at 
most $O(\beta n)$. \\


THEOREM 6.2 Given $r$, $d=4k$, $k \geq r \geq 2$, there exist a positive 
constant $\epsilon = \epsilon(r)$ so that in {\color{blue}every} set of
more than $(2 - \epsilon(r))^{d}$ $(\pm 1)$-vectors there are $r$ 
pairwise orthogonal vectors.\\
%------------------------------------------------------------------------

\end{document}
































