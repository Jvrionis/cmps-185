
% James Vrionis



\documentclass[english]{uspatent}
\begin{document}

\setAssigneeName{University of California, Santa Cruz}
\setAssigneeAddress{1156 High Street}
\setAssigneeCity{Santa Cruz}
\setAssigneePhone{(831) 459-0111}
\setDocketNumber{US996285665Z9}
\setLawyerName{Mr. Madeu Pthisname}
%\setLawyerNumber{Patent Lawer Reg. Number}
%\setLawyerPhone{Patent Lawyer Phone}
%\setOtherInventor{Another Inventor}
%\setOtherInventor{Yet Another Inventor}
\setDocumentVersion{0.0}
\setPrintingModeApplication

%\include{Drawings}

\title{3D2D Transformative Spectacles}
\date{Date of this version}
\inventor{James Vrionis}

\maketitle

\patentSection{Field of the Invention}

\patentParagraph The invention relates to an image display system, and more particularly to the ability to  deflate a 3D aspect ratio back into a 2D image. 

\patentParagraph Optical device for real time translation of 3D images into 2D. 
Real time image translation apparatus made to adapt ordinary bifocal lenses 
to view 3D movies as a high definition 2D film. 

% In other words, the basic types of things that the invention 
% improves or is implemented in.  


\patentSection{Background of the Invention}

\patentParagraph Many IMAX movies are shown only in 3D and for many viewers not ready for 
Such an impressive experience, a need for this system was born. A 2D image system
on dedicated glasses worn by a viewer will now present the image as if it were never
3D. 

\patentParagraph Conventional 3D passive image processing frames, while quite functional, are often not very useful to those that want a nice 2D image. Although the experience may be great, they may want a less in your face, intrusive view. Unfortunately, conventional lenses only show
enhance an image which to a 3D perspective, so that it becomes difficult or impossible to even enjoy such an expierence because its doesnt exist. 

%\patentParagraph The things that are needed will be put forth as solutions in the next section.

\patentSection{Objects of the Invention}

\patentParagraph It is an object of this invention to tranlate 3D images shown at many
IMAX theatres back to 2D. It will not dimisish the quality or expierence the user or those
arround the user. 

\patentParagraph Necessary materials: Plastic, polarizing and nonpolarizing film.

\patentParagraph Still other objects and advantages of the invention will in part be obvious and will in part be apparent from the specification and drawings.

\patentSection{Prior Art}

\patentParagraph US20130009947A1 Passive 3D 

\patentParagraph US20120314937A1 Method and apparatus for providing a multi-view still image 
service

\patentSection{Summary of the Invention}

\patentParagraph In order to overcome 3D images, we tranlate them into 2D images. 

% \patentParagraph The invention accordingly comprises the several steps and the relation of one or more of such steps with respect to each of the others, and the apparatus embodying features of construction, combinations of elements and arrangement of parts that are adapted to affect such steps, all is exemplified in the following detailed disclosure, and the scope of the invention will be indicated in the claims.

%\patentDrawingDescriptions

\patentSection{Detailed Description of the Preferred Embodiments}

\patentParagraph The details of the invention go here. I will use this area to make reference to the drawings so you can see how it's done.

\patentParagraph The arrangement in reference to this object will contain three different flims,
bonded together in order to un3D this image. A layer of polarizing film will be sandwiched
in-between two nonpolarizing sheets. Taking these lenses and bonding them with a frame that 
comfortably places said lenses infront of each eye allowing any user to enjoy 2D at any time. 
%\referencePatentFigure{VisioDrawing} shows an exemplary 
%arrangement of a preferred embodiment. In \referencePatentFigure{VisioDrawing}, one sees a \annotateWithName{Widget} and a \annotateWithName{Thing} with a preferable \annotateWithName{WidgetThingConnection} that enables the \annotateWithName{Thing} to process the data coming from the \annotateWithName{Widget}. I think you get the idea.  You can refer to the number as \annotationNumberReference{Widget} and if you need it underlined in a drawing, use \annotationNumberReferenceUnderlined{Widget}.

% \patentParagraph You can either write: \annotateWithName{Thing} or you can write thing~\annotate{Thing}. They both produce the same thing.

% \patentParagraph Note that you make and refer to equations like this:

%\begin{equation}
%E=mc^{2}\label{eq:energy}
%\end{equation}

%\patentParagraph One of my favorite equations is:

%\begin{equation}
%e^{j\theta}=\cos\left(\theta\right)+j\cdot\sin\left(\theta\right)\label{eq:euler}
%\end{equation}

% \patentParagraph We refer to the first equation as \prettyref{eq:energy} and the second as \prettyref{eq:euler}. The second equation \prettyref{eq:euler} is Euler's equation.

% \patentParagraph It will thus be seen that the objects set forth above, among those made apparent from the preceding description, are efficiently attained and, because certain changes may be made in carrying out the above method and in the construction(s) set forth without departing from the spirit and scope of the invention, it is intended that all matter contained in the above description and shown in the accompanying drawings shall be interpreted as illustrative and not in a limiting sense.

% \patentParagraph It is also to be understood that the following claims are intended to cover all of the generic and specific features of the invention herein described and all statements of the scope of the invention which, as a matter of language, might be said to fall therebetween.

% im here 

\patentClaimsStart

\beginClaim{Claim1}
An optical 2D image system, for receiving a passive 3D image  comprising N rows of modified data for this 2D image system comprising: \\
\begin{itemize}[label={}]
\item A means of 2D image display module, comprising nonpolarizing film to polarizing film back to nonpolarizing film; 

\item A descaling module, for dilation for N/2 rows of scaled data in N rows back to original aspect ratio by 2N; 
\item A luminance readjusting module, for re-adjusting N/2 rows of scaled data to decrease a luminance of the scaled image;

\item wherein, the descaling module comprises N/2 rows of reprocessed data, and is displayed by the optical 2D image display system. \\
\end{itemize}

\beginClaim{Claim2}
  
A optical 2D image system according \claimRef{Claim1} wherein this system is surround by a 
semi-fexible plastic frame. 

\patentClaimsEnd

\patentSection{Abstract}
An optical 2D image system for receiving a 3D image that have N rows of modified rows of data back to a 2D image. This optical system dilates N/2 rows of scaled data to 2N from a left-eye image or a right-eye image. The image will be readjusted back to its original aspect ratio.

%\patentDrawings
\end{document}
