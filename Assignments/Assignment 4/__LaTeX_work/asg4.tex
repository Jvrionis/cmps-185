\documentclass[11pt]{article}
%\usepackage{times}
\usepackage{fancyhdr, enumitem, amsmath, amssymb, amsthm}
\usepackage[svgnames]{xcolor}
%\usepackage[a4paper]{geometry}
\usepackage[margin=1in]{geometry}
\usepackage{url, hyperref} 
\usepackage{fullpage}
%\setlength{\parskip}{5pt plus 1pt}
\setlength{\headheight}{13.6pt}
\newcommand\question[2]{\vspace{.25in}\hrule\textbf{#1: #2}\vspace{.5em}\hrule\vspace{.10in}}

\newcommand\ABIT{\vspace{.10in}\textbf{About: \\}}
\newcommand\PUBLH{\vspace{.10in}\textbf{Publishing: \\ }}
\newcommand\PAGEL{\vspace{.10in}\textbf{Page limits of submitted articles: \\ }}
\newcommand\REVIEW{\vspace{.10in}\textbf{Review Policy: \\ }}
\newcommand\OAS{\vspace{.10in}\textbf{Open Access: \\ }}
\newcommand\EINFO{\vspace{.10in}\textbf{Extra Info:  \\ }}
\newcommand\LGRADE{\vspace{.10in}\textbf{Letter Grade (A-D):  }} \\

%\newcommand\ART1{\vspace{.10in}\textbf{Rasmus Pagh, "Large Scale similarity Joins with Guarentees" }}\\
%\newcommand\ART2{\vspace{.10in}\textbf{Wolfgang Lehnher, Next-Gen Hardware for Data Management More a bleesing than a Curse?"\\}}
%\newcommand\ART3{\vspace{.10in}\textbf{Graham Cormode "The confounding problem of private data release" \\ }}

\newcommand\PARTI{\vspace{.10in}\textbf{Part I \\}}
\newcommand\PARTII{\vspace{.10in}\textbf{Part II \\}}
\lhead{\textbf{\NAME\ (\JAMESID)}}  % Dont forget to change this 
\chead{\textbf{HW\HWNUM}}  
\rhead{\today}   
\begin{document}\raggedright
\newcommand\NAME{James Vrionis}  % name
\newcommand\JAMESID{JVrionis}     % Id 
\newcommand\HWNUM{4}    


\PARTI
\textbf{Name of Article:  }\textit{Autonomous Robots}\\ 
\textbf{Publisher:}Springer \\
\texittt{http://www.springer.com/engineering/control/journal/10514/PS2} \\
\textbf{Editor-in-Chief:} Gaurav Sukhatme \\
\textbf{Editorial Board size:} 12 members. \\
ISSN: 09295593 (print version) \\
ISSN:1573-7527 (electronic version) \\
Journal \#: 10514 \\

\ABIT 
Theory and applications of Robotic systems capable of some degree of self-sufficiency.
Main focus is on the ability to move and be self-sufficient. 

\PUBLH
How often is the journal published? 
1 volume with 8 issues per annual subscription. 

\PAGEL
Papers should be no longer than 75,000 character and this includes spaces. If a paper exceeds this limit it may be returned without a review. 

\REVIEW
According to the "Ethics \& Disclosures" section the journal publishes content via peer review and gives no implication as to double or single blind. 

\OAS
After you article has been accepted for publication, Spriner will email you with a link to MyPublication process page so you can decide to opt in for open access or not. 
Decision to publish open access cannot be changed after you have finished the MyPublication process. 
Open Choice is an available option via the payment for an open access publication fee of \$ 3000US or 2200 EUR. 

\EINFO
Control of autonomous robots, Real-time vision, Autonomous wheeled and tracked vehicles, Legged vehicles, Computational 
architectures for autonomous systems, Distributed architectures for learning: control and adaptation, 
Studies of autonomous robot systems, Sensor fusion, Theory of autonomous systems,
Terrain Mapping and recognition, Self-calibration and self-repair for robots, 
self-reproducing intelligent structure, and Genetic algorithms as models for robot development. 


How to Submit and Review: \texittt{http://www.editorialmanager.com/auro/default.aspx}\\

  
%---------------------------------------------------------------------------------------------------------------------
\newpage


\textbf{Name of Article:  } ACM Journal on Emerging Technologies in Computing Systems (JETC) \\
\textbf{Publisher:} Association for Computing Machinery (ACM) Digital Library \\
\texittt{https://dl.acm.org/pub.cfm?id=J967}  and  \texittt{https://jetc.acm.org}\\
\textbf{Editor-In-Chief:}  Yuan Xie, University of California, Santa Barbara \\
\textbf{Editorial Board Size:} 19 members \\


\PUBLH
Seems like a new volume is published when there is enough information to fill a issue. It doesn't look 
like there is any pattern to the publication. List of publications (ongoing..): \\

\textbf{2018:} Volume 14 Issue 1, January 2018 (Issue-in-Progress)   \\
\textbf{2017:} Volume 13 Issue 4, August 2017,  Volume 13 Issue 3, \\
May 2017 (Special Issue on Hardware and Algorithms for Learning On-a-chip and Special Issue on Alternative Computing Systems) \\
Volume 13 Issue 2, March 2017   (Special Issue on Nano electronic Circuit and System Design Methods for the Mobile Computing Era and Regular Papers) \\
\textbf{2016:} Volume 13 Issue 1, December 2016 (Special Issue on Secure and Trustworthy Computing)
Volume 12 Issue 4, July 2016 (Regular Papers) \\
\textbf{2015:} Volume 12 Issue 3, September 2015 (Special Issue on Cross-Layer System Design and Regular Papers)
Volume 12 Issue 2, August 2015  (Special Issue on Advances in Design of Ultra-Low Power Circuits and Systems in Emerging Technologies) Volume 12 Issue 1, July 2015 Volume 11 Issue 4, April 2015 (Special Issues on Neuromorphic Computing and Emerging Many-Core Systems for Exascale Computing) \\
\textbf{2014:} Volume 11 Issue 3, December 2014 (Special Issue on Computational Synthetic Biology and Regular Papers),
Volume 11 Issue 2, November 2014 (Special Issue on Reversible Computation and Regular Papers), Volume 11 Issue 1, September 2014, Volume 10 Issue 4, May 2014, Volume 10 Issue 3, April 2014, Volume 10 Issue 2, February 2014, Volume 10 Issue 1, January 2014 (Special Issue on Reliability and Device Degradation in Emerging Technologies and Special Issue on WoSAR 2011) \\
\textbf{2013:} Volume 9 Issue 4, November 2013 (Special Issue on Bioinformatics), Volume 9 Issue 3, September 2013, Volume 9 Issue 2, May 2013 (Special issue on memory technologies), Volume 9 Issue 1, February 2013  \\
\textbf{2012:} Volume 8 Issue 4, October 2012, Volume 8 Issue 3, August 2012, Volume 8 Issue 2, June 2012 (Special Issue on Implantable Electronics), Volume 8 Issue 1, February 2012   \\
\textbf{2011:} Volume 7 Issue 4, December 2011, Volume 7 Issue 3, August 2011, Volume 7 Issue 2, June 2011, Volume 7 Issue 1, January 2011   \\
\textbf{2010:} Volume 6 Issue 4, December 2010, Volume 6 Issue 3, August 2010, Volume 6 Issue 2, June 2010, Volume 6 Issue 1, March 2010    \\
\textbf{2009:} Volume 5 Issue 4, November 2009, Volume 5 Issue 3, August 2009, Volume 5 Issue 2, July 2009, Volume 5 Issue 1, January 2009  \\
\textbf{2008:} Volume 4 Issue 4, October 2008, Volume 4 Issue 3, August 2008, Volume 4 Issue 2, April 2008, Volume 4 Issue 1, March 2008, Volume 3 Issue 4, January 2008    \\
\textbf{2007:} Volume 3 Issue 3, November 2007, Volume 3 Issue 2, July 2007, Volume 3 Issue 1, April 2007  \\ 
\textbf{2006:} Volume 2 Issue 4, October 2006   , Volume 2 Issue 3, July 2006, Volume 2 Issue 2, April 2006, Volume 2 Issue 1, January 2006  \\
\textbf{2005:} Volume 1 Issue 3, October 2005, Volume 1 Issue 2, July 2005, Volume 1 Issue 1, April 2005 \\

\PAGEL 
Publishes research paper no more than 25 pages in ACM journal/Transactions format and
For tutorial and survey papers a strict limit of 40-50 pages with an extensive bibliography.\\


\REVIEW
Papers submitted to this journal are to be evaluated by anonymous referees (to the author). 
The referees will be notified of the name of an associate editor who knows the identity of
both, the author and the referee. The final point of acceptance lies with the Editor-In-Chief 
who may either reject or accept any manuscript. \\


\OAS
New author-pays option must be chosen for open acess via ACM Digital Library. 
Fees for Open Access are applied only to full papers and are as follows:\\

\begin{tabular}{ | l | c | r |} 
\hline
\mbox{Authors} & \mbox{No ACM or SIG members} & \mbox{At least 1 ACM or SIG member}\\
\hline
\mbox{Journal Article} & \$ 1700 & \$ 1300 \\
\hline
\mbox{Proceedings Articl}e & \$ 900 & \$ 700 \\
\hline
\mbox{Proceedings of ACM Article} & \$ 900 & \$ 700 \\
\hline 
\end{tabular}

\EINFO
ACM Author rights allow three different ways for publication management. First, 
a non-exclusive permission to publish recieived via purchase of the perpetual 
open access from the ACM Digital Library. This allows authors the ability to 
self-manage their work. Second would be a new Publishing License Agreement that 
grants ACM the right to serve as the exclusive publisher of said work , with the 
ability to defend it against improper use by third parties. This second option 
could be seen as an equivalent to ACM's traditional Copyright Transfer Agreement 
except that the author holds the copyright. Lastly, ACM's traditional Copy 
Transfer Agreement. More information is available here 
\texittt{http://authors.acm.org} \\



\PARTII \\


\newline

\textbf{Rasmus Pagh, "Large Scale similarity Joins with Guarentees" } \\
\LGRADE A \\
This article deserved an A compared to the other two, It had a consistent color scheme,
good flow of information, catchy titles, and a similiar font-seize throughout. This
keynote must have been a great presentation! I like how Pagh ended with a quote by Monty Python, 
"There's nothing an agnostic can't do if he doesn't know whether he believes in it or not." 
In my opinion this is a nice way to end while still keeping the audience engaged. \\ 
one-hundred and twelve slides seems like alot but in could be okay if you take into account
the amount of usefull information Pagh put on each slide accompanied it would be fine. I 
didn't notice any exit strategies or checkpoints but it was lightyears ahead of the other two.
Just as J. Zobel has said, "Talks are inherently linear" and reinforcement would be very 
important, especially in a talk as long as this one. \\

\newpage

\textbf{Wolfgang Lehnher, Next-Gen Hardware for Data Management More a bleesing than a Curse?"}\\
\LGRADE C- \\

There seems to be many different colors of font with more than 2 different font sizes. This would violate our 
two font sizes per slide rule and our use color sparingly rule.\\
On the Evaluation: Some MicroBenchmarking slide the colors used are very distasteful. 
The title uses a turquoise and the two graphs use a Kermit the frog green contrasted to a red. 
I don't like the Lehnher color choices. \\
I also don't like the use of multiple graphs on one slide. It makes the information very distracting. 
It may be better during a presentation with Lehnher directing me through it but as it is I do not like it.
The slides seem very loaded and feel as if Lehnher would benefit greatly by blowing the size of the graphs 
up and keeping to our one per page max. 
This presentation lacks checkpoints and a roadmap but isn't \textit{horrible}. \\

\textbf{Graham Cormode "The confounding problem of private data release" }\\
\LGRADE C+ \\
Lastly this keynote has horrible typesetting and font color in title on slide 3, Lacks a roadmap and checkpoints.
Colors on slide 10 are too much. It distracts from main idea. It lacks a good flow (not bad, but not good).\\ 
Too many colors on slide 17 when explaining Differential Privacy formula. Slide 18 could have been broken down 
in three slides for readability. Slide 19 could also be broken down in two slides for readability.
Color choices are a common annoyance horrible graph layout and units. Slides 26,27,28 should all be spread out and 
adequately labeled and displayed better ascetically. Citing [SIGMOD] in the title on a slide left a bad taste in my mouth.\\


\textbf{I will finish editing and formating for next turn in. had alot of work needed to turn this in.}\\



\end{document}