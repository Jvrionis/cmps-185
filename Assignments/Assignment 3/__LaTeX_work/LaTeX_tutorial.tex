% %---------------------- begin preamble ----------------------
% % Modify the article document class, so that it 
% %  produces two columns per page in 11pt
% \documentclass[11pt,twocolumn]{article}
% %\usepackage{times}
% \usepackage[square]{natbib} 
% \usepackage{enumitem, amsmath,amssymb,amsthm}
% \usepackage[a4paper]{geometry}
% \usepackage{url, hyperref} 
% \usepackage{fullpage}
% \usepackage[svgnames]{xcolor}
% \usepackage{tikz}
% \usetikzlibrary{automata, positioning, arrows}

% %ceated during my cmps102  class.
% \renewcommand\part[1]{\vspace{.10in}\textbf{(#1)}}
% \newcommand\DALGORITHM{\vspace{.10in}\textbf{Description of Algorithm: }}
% \newcommand\PCORRECT{\vspace{.10in}\textbf{Proof of Correctness: }}
% \newcommand\TIME{\vspace{.10in}\textbf{Analysis of Time: }}
% \newcommand\SPACE{\vspace{.10in}\textbf{Analysis of Space: }}

% %----------------------------------------------------------------------
% % 5-7pgs
% %
% % Document classes (describe at least three).
% % How to make a title and a list of authors and their affiliations
% % (this should go beyond what was discussed in class).
% % Sections, subsections, sub-subsections, paragraphs, and their 
% %  labeling.
% % Environments, including the tabular environment (describe at 
% %  least three environments).
% % Mathematical formulas.
% % User-defined macros (give at least three interesting examples, all 
% %  different from the ones discussed in class).
% % Bibliography and citation.
% % At least one additional feature of LATEX that you find worth 
% %  including in your tutorial (e.g.,creating figures, graphing data, 
% %  making an index for a thesis or book).
% %
% % Your document should contain a bibliography listing a few 
% %  references with information about
% %----------------------------------------------------------------------

% %----------------------- end preamble -----------------------
% \begin{document}
% \title{Basic \LaTeX{}: Features Explained} \\
% \author{James M. Vrionis\\
%         CMPS-185 \\
%         University of California, Santa Cruz
%         \date{1-26-2018} 
%         %\texttt{}
%        }
% \maketitle 

% %----------------------------------------------------------------------
% ~\cite{wiki_tables}

% \section{Introduction} 
% Pronounced either LAH-tek or LAY-tek, 
% \LaTeX{} is a \textit{free} typeseting language 
% developed by \textit{Leslie Lamport}  
% based on \textit{Donald E. Knuth's}  
% \TeX{}.\\

% Markup Language Key Features: ~\cite{rjgunning_gunning_2015} 
% \begin{itemize}[label={}]
%    \item Typesetting journal articles.
%    \item Technical reports. 
%    \item Bibliographies and indexes.
%    \item and many more.
% \end{itemize}

% \section{Install}
% \subsubsection{Ubuntu Linux}
% Open Terminal to install \textbf{TEX LIVE} via the \textit{apt-get}
% command: \\
% % Visual of where the preamble is in latex document
% \framebox{
% \parbox[l][1.50cm]{6.0cm}{
% \addvspace{0.2cm} \centering 
% \textit{sudo apt-get install texlive-full} \\
% } 
% }\\

% \textit{NOTE: it's always good to perform a} 
% \textbf{"sudo apt-get upgrade \&\& sudo apt-get update"} 
% \textit{before installing anything new.}\\

% now ".tex files" can be compiled using the pdflatex $<$your file$>$.tex
% as long as you in the same directory as the file. \\


% \begin{center}
% Alternatively, you may install from the link below \\
% \textittt{http://www.tug.org/texlive/}\\
% \end{center}

% \subsubsection{MacOS}
% Open Terminal to install \MacTeX{} using \textit{Homebrew Cast}.\\

% \framebox{
% \parbox[l][2.50cm]{6.5cm}{
% \addvspace{0.2cm} \centering 
% \textit{macbk-usr\$ brew tap caskromm/cask} \\
% \textit{macbk-usr\$ brew cask install mactex} \\
% } 
% }\\

% now ".tex files" can be compiled using the pdflatex $<$your file$>$.tex
% as long as you in the same directory as the file.\\

% \begin{center}
% Alternatively, you may install from the link below \\
% \textittt{http://www.tug.org/mactex/}\\
% \end{center}

% \subsubsection{Windows}
% To install and use a \LaTeX{} distribution on Windows 
% operating system one should install \textbf{\TeX{} Live} 
% via the installer.\\


% \begin{center}
% \textbf{Install via link below}\\
% \texittt{http://www.tug.org/texlive/} 
% \end{center}


% \section{Compilation}

% All three Operating Systems have very similar terminal
% command line arguments necessary to compile you ".tex"
% file $\to$ the most commonly desired ".pdf".\\
% In an open terminal simply type the following:\\
% 'pdflatex you\_file\_name.tex' and within the current
% working directory '.pdf' file will be created.\\

% Linking '.bib' files take alittle more work. I will
% explain the steps as follows for a file named 
% blah.tex. \\ 

% If you have not already done so, change the current
% directory of terminal to that which contains blah.tex
% as well as another file created by you called blah.bib.
% As long as there are not any errors within you latex
% code 4 commands make this possible:\\
% pdflatex blah.tex \\
% bibtex blah.aux \\
% pdflatex blah.tex \\
% pdflatex blah.tex\\

% It is important to execute the pdflatex command twice
% after bibtex blah.aux has been run.
% This will created your desired pdf-file with an added
% references section (assuming you have entries in the
% .bib file). The last section of this document will 
% contain more information about bibliographies. \\


% %----------------------------------------------------------------------
% % At least 3 Document Class explanations
% \section{Document Classes} 
% The \textbf{Preamble} in a \LaTeX{} 
% document is defined as the area prior to any invocation of 
% begin/end \{document\} tags. This space is used to define the 
% type of document class you will be using, the page layout desired,
% and user defined commands.\\ 

% % Visual of where the preamble is in latex document
% \framebox{
% \parbox[t][3.0cm]{5.50cm}{
% \addvspace{0.2cm} \centering 
% \framebox{
% \parbox[t][1.0cm]{4.50cm}{
% \addvspace{0.2cm} \centering 
% \textbig{THE}  \textbig{PREAMBLE} \\
% } 
% }\\
% \textbf{ \textbackslash{begin}\{document\}} \\ 
% \textbf{ \vdots }  \\
% \textbf{ \textbackslash{end}\{document\} }
% } 
% }\\

% This is where your \textit{Document type} is to be defined. \\

% \textbf{Types of Document Classes}\\
% There are nine main \LaTeX{} defined textit{Document Classes}.
% They are as follows: \\
% {\color{blue} \textbf{article}},
% {\color{blue} \textbf{IEEEtran}},
% {\color{blue} \textbf{proc}},
% {\color{blue} \textbf{report}}, 
% {\color{blue} \textbf{book}}, 
% {\color{blue} \textbf{slides}}, 
% {\color{blue} \textbf{memior}}, 
% {\color{blue} \textbf{letter}}, and 
% {\color{blue} \textbf{beamer}}.\\

% In order to use these one of these \textit{Document Classes} one must
% use the following command: 
% \textbackslash{documentclass\{Document-Class\}}. \\ 
% where \textit{Document-class} must be replaced with one of the nine
% blue labeled document classes (show above). \\
% Each of these have their own unique advantages and one package may
% be already optimized for your particular use, but to use any of these 
% environments they must be placed with-in the \textbf{preamble} 
% of your document. \\

% % The \textbf{Preamble} in a \LaTeX{} 
% % document is defined as the area prior to any invocation of 
% % begin/end \{document\} tags. This space is used to define the 
% % type of document class you will be using, the page layout desired,
% % and user defined commands.\\ 

% %This is where your ~\textit{Document type} is to be defined.


% % The \textbf{preamble} is defined
% % as the area before you have started your document. That is done 
% % using begin and end document commands. \\

% %\textbackslash{begin\{ document \}}.
% %----------------------------------------------------------------------
% % Define article: 
% % a piece of writing included with others in a newspaper, magazine, or 
% % other publication
% \subsection{Article}
% %description:
% \begin{itemize}
%    \item \textbf{Description:}
%       \begin{itemize}[label={}]
%          \item Quite possibly the most popular \LaTeX{} class used is 
%          textbf{article}. 
%          \item It is mainly used for short documents and 
%          journal articles it can also be used for the following: 
%          articles in scientific journals, presentations, short reports,
%          program documentation, invitations, homework, note-taking, 
%          and many more.
%       \end{itemize}
%    \item \textbf{Commands:}
%       \begin{itemize}[label={}]
%       \item \textbackslash{part\{\}}
%       \item \textbackslash{chapter\{\}}
%       \item \textbackslash{section\{\}}
%       \item \textbackslash{paragraph\{\}} 
%       \item \textbackslash{subsection\{\}}
%       \end{itemize} ~\cite{latex_wikia}
% %How to use:
%    \item \textbf{How to use:}
%       \begin{itemize}
%       \item To use this particular document class within your \LaTeX{} 
%        document, the \textbackslash{documentclass\{article\}} in the 
%        area defined as the preamble.
%       \end{itemize}
% \end{itemize}
% %----------------------------------------------------------------------
% ~\cite{gmurray_sbalemi}
% \subsection{IEEEtran} 
% %description:
% \begin{itemize}
%    \item \textbf{Description:}
%       \begin{itemize}
%       \item Institute of Electrical and Electronics Engineers (IEEE)
%       transactions journals and conferences is what is meant by 
%       by IEEEtran document class name. This document class is 
%       particularly useful for (IEEE) authors and has many useful
%       commands and layout required by this organization.
%       \end{itemize}
%    \item \textbf{Commands:}
%       \begin{itemize}[label={}]
%       \item \textbackslash{thanks\{\}}
%       \item \textbackslash{abstract\{\}}
%       \item \textbackslash{bibliography\{\}}
%       \item \textbackslash{Theorem\{\}} 
%       \item \textbackslash{cite\{\}}  
%       \end{itemize}
% %How to use:
%    \item \textbf{How to use:}
%       \begin{itemize}
%       \item To use this particular document class within your \LaTeX{} 
%        document, the \textbackslash{documentclass\{IEEEtran\}} in the 
%        area defined as the preamble.
%       \end{itemize}
% \end{itemize}
% %----------------------------------------------------------------------

% %----------------------------------------------------------------------
% \subsection{book}
% %description:
% \begin{itemize}
%    \item \textbf{Description of Book:}
%       \begin{itemize}
%       \item The structure of this document class
%       contains "chapter" which a header 
%       automatically made with pages and the name
%       of section on odd pages. Pages are always
%       2-sided by default.  
%       \end{itemize}
%    \item \textbf{Commands:}
%       \begin{itemize}[label={}]
%       \item \textbackslash{part\{\}}
%       \item \textbackslash{chapter\{\}}
%       \item \textbackslash{section\{\}}
%       \item \textbackslash{paragraph\{\}} ~\cite{latex_wikia}
%       \item \textbackslash{subsection\{\}}  
%       \end{itemize}
% %How to use:
%    \item \textbf{How to use:}
%       \begin{itemize}
%       \item To use this particular document class within your \LaTeX{} 
%        document, the \textbackslash{documentclass\{book\}} in the area
%        defined as the preamble.
%       \end{itemize}
% \end{itemize}
% %----------------------------------------------------------------------


% %----------------------------------------------------------------------
% % Go beyond what was discussed in class
% \section{Title, Listing Multiple \\ Authors and Affiliations}

% Titles in \LaTeX{} can be made using \textbackslash{maketitle} or
% by using the \textbf{titlepage} environment. As a little example,
% here is the code I wrote to make the title of this document:\\

% \flushleft{\textbackslash{title}\{Basic\textbackslash{LaTeX\{\}}: Features Explained \} 
% \textbackslash \textbackslash }\\
% \textbackslash{author} \{ James M. Vrionis \textbackslash \textbackslash \\
%          CMPS-185, \textbackslash \textbackslash \\
%          University of California, Santa Cruz \textbackslash \textbackslash \\
%       \} \textbackslash \textbackslash \\
% \textbackslash{maketitle}\\

% My title was set up completely using the 
% \textbf{maketitle} as you can see
% it outputs a very professional title.


% \subsection{Title}
% Titles in \LaTeX{} can be made using \textbackslash{maketitle} or
% by using the \textbf{titlepage} environment. As a little example,
% here is the code I wrote to make the title of this document:\\

% \textbackslash{title}\{Basic \textbackslash{LaTeX\{\}}: Features Explained \} 
% \textbackslash \textbackslash \\
% \textbackslash{author} \{ James M. Vrionis \textbackslash \textbackslash \\
%          CMPS-185, \textbackslash \textbackslash \\
%          University of California, Santa Cruz \textbackslash \textbackslash \\
%       \} \textbackslash \textbackslash \\
% \textbackslash{maketitle}\\

% This code would be placed imediately after \textbackslash{begin\{document\}}.

% \subsubsection{Title Page}
% Another way to accomplish a \textbf{title} 
% would be to make a \textbf{title page}. 
% Another example should make the use of this clearer,
% so first we need to define attributes within the
% preamble portion of your document.\\ 

% \textbackslash{title}{Basic \textbackslash{LaTeX\{\}}: Features Explained} \\
% \textbackslash{author}James M. Vrionis \thanks{Inpired by CMPS-185} \\
% \textbackslash{date}{January 27,  2018} \\

% Next, imediately after our \textbackslash{begin}document tag 
% we can put the do the following: 

% % Visual of where the preamble is in latex document
% \framebox{
% \parbox[t][6.0cm]{5.50cm}{
% \addvspace{0.3cm} \centering 
% \framebox{
% \parbox[t][3.50cm]{4.50cm}{
% \addvspace{0.2cm} \centering 
% \textbackslash{title}\{Basic \textbackslash{LaTeX\{\}}: Features Explained\} \\
% \textbackslash{author}\{James M. Vrionis \textbackslash{thanks}\{Inpired by CMPS-185\}\} \\
% \textbackslash{date}\{January 27,  2018\} \\
% } 
% }\\
% \textbf{ \textbackslash{begin}\{document\}} \\ 
% \textbackslash{begin}\{titlepage\}  \\
% \textbackslash{maketitle}\\
% \textbf{ \textbackslash{end}\{titlepage\} }
% } 
% }\\

% All attributes of the \textbf{Title} should be defined 
% within the preamble and like we have seen before 


% \subsection{Multiple Authors and Affiliations}
% To list multiple Authors and Affiliations the 
% \textbackslash{author} command will suffice.
% Let me show you another nice example of how this 
% can be accomplished: \\

% \textbackslash{author}\{ \\
%    Author 1 \textbackslash \textbackslash \textbackslash \\
%    Company or School \textbackslash \textbackslash \\
%    $\cdots$ etc. $\cdots$ \textbackslash \textbackslash \\
%    \textbackslash{and} \\
%    Author 2 \textbackslash{thanks{$\cdots$}} \textbackslash \textbackslash \\
%    $\cdots$ etc. $\cdots$ \textbackslash \textbackslash \\
%    \} \\
% \textbackslash{date} \textbackslash \textbackslash \\
% \textbackslash{maketitle} \\

% %----------------------------------------------------------------------


% %----------------------------------------------------------------------
% \section{Sections}
% These are the \LaTeX{} commands very specific to the organization 
% of document types. A section creates a numbered title above text 
% its associated block of material. The numbering system is set by 
% default but is easily turned off by placing an asterisk before the 
% first opening curly brace. \footnote{ Using the asterisk will remove 
% this section from a Table of Contents and you must re-add, 
% via \textbackslash{addcontentsline} placed to the right of  
% \textbackslash{section}$\asterisk$\{SoMe\_SecTioN\} \\


% By default, sections are numbered and in bold font. Say we are looking
% some text and we come across bolded text \textbf{5 Sections} above the 
% text we are reading, which means section 5 of our text and
% anything below it, whether it be a subsection, subsubsection or 
% paragraph should be related to this main idea.


% \framebox{
% \parbox[c][7.50cm]{5.50cm}{
% \begin{itemize}[label={}]
%    \item \textbf{Section 1}
%    \begin{itemize}[label={}]   
%       \item Paragraphs
%       \item Subsection 1
%       \begin{itemize}[label={}]
%          \item Paragraphs
%          \item Sub-subsection 1 
%          \begin{itemize}[label={}]
%             \item Paragraphs
%          \end{itemize}
%          \item Sub-subsection 2 
%          \begin{itemize}[label={}]
%             \item Paragraphs
%          \end{itemize}
%       \end{itemize}
%       \item Subsection 2
%    \end{itemize}
%    \item \textbf{Section 2}
% \end{itemize}
% } 
% }\\

% and so on...\\ 

% \textbackslash{section}\{ \textcolor{blue}{Title of Section\}}} \\

% \subsection{Subsections}
% A \textbf{\textbackslash{\{subsection\}}} is a child of its 
% parent section. If you are in section 4.0, the following 
% subsections would be 4.1, 4.2,..., 4.x where x can be a any 
% number larger than its predecessor.  \\

% If you are in section 5.0, the following subsections would be 
% 5.1, 5.2,...,5. {\color{blue} x} where {\color{blue} x} can be any 
% number larger than its predecessor. An easy interpretation would be 
% {\color{red} Section}.{\color{blue} Subsection}\\

% \subsection{Sub-subsections}
% A \textbf{\textbackslash{\{subsubsection\}}} is the grandchild its some 
% preceding parent section. 5.1.1 would read: section 5, subsection 1 
% of subsubsection 1. Possible sub-subsections are 
% 5.1.1, 5.1.2, ...,5.1.{\color{DarkGreen}x}, with {\color{DarkGreen}x} to any 
% number larger than its predecessor.\\ 
% An easy interpretation would be {\color{red} Section}.
% {\color{blue} Subsection}.{\color{DarkGreen} Sub-subsection}\\

% \subsection{Paragraphs} ~\cite{latex_wikia} 
% When using the \textbf{\textbackslash{begin}\{paragraph\}} 
% command text is broken down into lines, 
% and lines are broken down into pages. To end a 
% paragraph there must be one or more blank lines 
% following the text inside the paragraph.\\ 

% Your text should take place of the vertical dots.\\
% % Visual of where the preamble is in latex document
% \framebox{
%     \parbox[c][2.0cm]{3.50cm}{
%         \addvspace{0.2cm} \centering 
%         \textbf{\textbackslash{begin}\{paragraph\}}\\ 
%         \textbf{  \vdots}\\
%         \textbf{\textbackslash{end}\{paragraph\}}
%     } 
% }\\

% \subsection{Labeling} ~\cite{wiki_books_labels_cross}
% Labels in \LaTeX{} are references to previous place in your 
% document enabling a much better way to point back to previous 
% marked objects. \\

% Objects are marked with \textbackslash{label}\{ some\_name \} 
% and can be retrieved using \textbackslash{ref}\{ some\_name \} 
% command. \\ 

% \framebox{
%     \parbox[c][3.0cm]{4.50cm}{
%         \addvspace{0.2cm} \centering 
%         \textbackslash{\{section\}}
%         \textbf{\textbackslash{label\{\textcolor{red}{labbbel}\}}}\\ 
%         \textbf{ \vdots }\\
%         \textbf{\textbackslash{ref\{\textcolor{red}{labbbel}\}}}
%     } 
% }\\

% %----------------------------------------------------------------------


% %----------------------------------------------------------------------
% % Describe at least three environments including tabular
% % Computer code, no matter what special characters it has, may be 
% % listed with the verbatim environment:
% \section{Environments} ~\cite{wiki_tables}
% \LaTeX{} comes equipt with many different environments. Three 
% environments we will look at are {\color{blue}\textbf{Tabular}},
% {\color{blue}\textbf{Itemize/Enumerate}}, and 
% {\color{blue}\textbf{Description}}. ~\cite{wiki_tables} \\

% \subsection{Tabular} 
% \textbf{Tabular} is an allignment environment with vertical
% and horizontal line separating capabilities. When one or both 
% of these capabilities are enabled many different arrangements
% become available with different column orientations, justifactions,
% and unique line drawing instructions. ~\cite{aj_roberts}\\
% \mbox{}\\
% \mbox{}\\
% \textbf{Syntax:}\\  
% \textbackslash{begin}\{tabular\}[pos]\{table spec\} \\
%     \textbackslash{begin}\{tabular\} \{\textcolor{red}{A}\}\\
%         \&  \&  \textbackslash\textbackslash \\
%     \textcolor{blue}{B}    \&  \&  \textbackslash\textbackslash \\
%         \&  \&  \textbackslash\textbackslash \\
%     \textbackslash{end}\{tabular\}\\
    
%     \begin{enumerate}[label={\Alph*}]
%         \textcolor{red}{\item Here we have a place to define columns.}
%         \begin{itemize}[label={}]
%             \item l for left-justified 
%             \item c for centered
%             \item r for right-justified
%         \end{itemize}
%         \textcolor{blue}{\item Here we have the number of entries in the table.}
%     \end{enumerate}

% %---------------------------------------------------0
% % Examples.. 
% %---------------------------------------------------0
% \framebox{
%     \parbox[c][15.5cm]{9.50cm}{
%         %\addvspace{0.2cm} 
%         \centering
%         \textbf{No line Example:} \\ 
%         %\mbox{}\\

%         \begin{tabular}{l c r}
%         \mbox{this is}&\mbox{ table using} &\mbox{ mbox}\\
%         \mbox{L-just col}&\mbox{ center col}&\mbox{ R-just col}\\
%         \mbox{using }&\mbox{ tabular }&\mbox{ envir }\\
%         \end{tabular}\\
%         \mbox{}\\
%         \mbox{}\\

%         \textbf{Same example but with an added $|$: }\\
%         \mbox{}\\

%         \begin{tabular}{l | c r}
%         \mbox{this is}&\mbox{ table using}&\mbox{ mbox} \\
%         \mbox{L-just col}&\mbox{ center col}&\mbox{ R-just col}\\
%         \mbox{using }&\mbox{ tabular }&\mbox{ envir } \\
%         \end{tabular}\\ 
%         \mbox{}\\
%         \mbox{}\\

%         \textbf{Added another $|$ to previous:} \\
%         \mbox{}\\

%         \begin{tabular}{l | c | r}
%         \mbox{this is}&\mbox{ table using}&\mbox{ mbox} \\
%         \mbox{L-just col}&\mbox{ center col}&\mbox{ R-just col}\\
%         \mbox{using }&\mbox{ tabular }&\mbox{ envir } \\
%         \end{tabular}\\ 
%         \mbox{}\\
%         \mbox{}\\


%         \textbf{Adding a \textbackslash{hline} to previous}: \\
%         \mbox{}\\

%         \begin{tabular}{l | c | r}
%         \hline
%         \mbox{this is}&\mbox{ table using}&\mbox{ mbox} \\
%         \mbox{L-just col}&\mbox{ center col}&\mbox{ R-just col}\\
%         \mbox{using }&\mbox{ tabular }&\mbox{ envir } \\
%         \end{tabular}\\ 
%         \mbox{}\\
%         \mbox{}\\

%         \textbf{Added another \textbackslash{hline} to previous:} \\
%         \mbox{}\\

%         \begin{tabular}{l | c | r}
%         \hline
%         \mbox{this is}&\mbox{ table using}&\mbox{ mbox} \\ \hline
%         \mbox{L-just col}&\mbox{ center col}&\mbox{ R-just col}\\
%         \mbox{using }&\mbox{ tabular }&\mbox{ envir } \\
%         \end{tabular}\\  
%         \mbox{}\\
%         \mbox{}\\
%     } 
% }\\
% %----------------------------------------------------------------------



% %----------------------------------------------------------------------
% %\newpage
% \subsection{Itemize and Enumerate}
% \textbf{itemize} is unordered list making environment whereas
% the \textbf{Enumerate} environment is numerically ordered.\\
% \framebox{
%     \parbox[c][3.0cm]{4.50cm}{
%         \addvspace{0.2cm} \centering 
%         \textbf{\textbackslash{begin}\{{itemize}\}}\\ 
%         \par \textbackslash{item} "Text goes here" \\
%         \textbf{ \vdots }\\
%         \textbf{\textbackslash{end}\{{itemize}\}}
%     } 
% }\\

% \framebox{
%     \parbox[c][3.0cm]{4.50cm}{
%         \addvspace{0.2cm} \centering 
%         \textbf{\textbackslash{begin}\{{enumerate}\}}\\ 
%             \par \textbackslash{item} "Text goes here" \\
%         \textbf{ \vdots }\\
%         \textbf{\textbackslash{end}\{{enumerate}\}}
%     } 
% }\\


% \subsection{Description}
% \textbf{Description} is an ideal list making environment for definitions.

% \framebox{
%     \parbox[c][3.0cm]{4.50cm}{
%         \addvspace{0.2cm} \centering 
%         \textbackslash{begin}\{description\}
%         \textbackslash{item}[\textbf{Word}] \\
%             \par Here is the definition\\
%         \textbackslash{item}[\textbf{Word2}] \\
%             \par Here is the definition 2 \\
%         \textbackslash{end}\{description\}        
%     } 
% }\\

% %----------------------------------------------------------------------
% \section{Mathematical Formulas}

% \LaTeX{} has many different ways to write mathematical expressions. 
% Using either \textbf{inline} or one of the many different 
% \textit{display} modes. Displays are separate from paragraphs and 
% used as its own code block whereas inline expressions can be used
% with-in writing environments. \\

% \begin{itemize}[label={}]
%     \item \textbf{Inline:}
%     \begin{itemize}[label={\bullet}]
%          \item  \mbox{\$ math equation inline \$}
%          \item  \textbackslash{[ math equation inline }\textbackslash{]}
%          \item  \textbackslash{( math equation inline }\textbackslash{)}
%     \end{itemize}
%     \item \textbf{Display:}
%     \begin{itemize}[label={\bullet}]
%          \item \textbackslash{begin}\{equation\} \textbackslash{end}\{equation\}
%          \item \textbackslash{begin}\{displaymath\} \textbackslash{end}\{displaymath\}
%     \end{itemize}
% \end{itemize}\\
% \textbf{inline vs display mode:}\\
% \begin{equation}
%     \mbox{\textbackslash{begin}\{equation\}}
%     \sum_{i=0}^{\infty} \frac{n}{n+1}
%     \mbox{  \textbackslash{end}\{equation\}}    
% \end{equation}
% vs.
% This is inline: $\sum_{i=0}^{\infty} \frac{n}{n+1}$ \footnote{for inline place sound eq. in \$ \$} \\ 

% \textbf{code}: \\
% \textbackslash{sum}\_\{$i=0$\}\^\{\textbackslash{infty}\} \textbackslash{frac}\{$n$\}\{$n+1$\} \\


% %----------------------------------------------------------------------



% %---------------------------------------------------------------------- 
% \section{User-defined macros} ~\cite{uhedu_macros}

% Creating your own \textit{commands} via chaining together different
% commands. The definition of which, must be in the preamble section 
% of your \textbf{.tex} file. \\
% Another possible use would be to redefine an command already
% defined by latex. This may lead to serious problems unlike our final
% safer \textit{provide command} that defines the command 
% if and only if (IFF) the command doesn't exist, otherwise no 
% meaning is changed. 


% \subsection{New Commands}

% % \newcommand\DALGORITHM{\vspace{.10in}\textbf{Description of Algorithm: }}
% % \newcommand\PCORRECT{\vspace{.10in}\textbf{Proof of Correctness: }}
% % \newcommand\TIME{\vspace{.10in}\textbf{Analysis of Time: }}
% % \newcommand\SPACE{\vspace{.10in}\textbf{Analysis of Space: }}

% \textbackslash{newcommand}\textbackslash{DALGO}\{\textbackslash{vspace}\{.10in\}
% \textbackslash{textbf}\{Discription of Algorithm:\}\} \\

% I made this simple example while in cmps\-102 class. 
% The \textbf{new command} defined here is \textbackslash{DALGO}. \\
% \begin{itemize}
%     \item \textbackslash{DALGO} produces the text "Discription of Algorithm:"
%     \item \textbackslash{newcommand} says we are creating a new command, 
%     \item \textbackslash{DALGO} named "DALGO",
%     \item \textbackslash{vspace}\{.10in\} .10inches 
%             of vertical space between this command 
%             and anything follwing an instance of 
%             \textbackslash{DALGO},
%     \item \{Discription of Algorithm:\} which prints "Discription of Algorithm: 
% \end{itemize}

% This can be really useful when using the same material over and 
% over again. \LaTeX{} also gives users the ability to redefine 
% existing commands via the \textbackslash{renewcommand}.
% An example of how this can be accomplished is as follows: \\

% Original command is \textbackslash{part\{a\}}. the output of the 
% unmodified command is\\ \textbf{Part I}\\ \textbf{a}\\
% Now, our after redefining this command we get the following 
% output:
% \part{a} "with text capabilities" imediately following the 
% command.  \\

% \framebox{
%     \parbox[c][3.0cm]{4.50cm}{
%         \addvspace{0.2cm} \centering 
%         \textbf{Original Command }\\
%         \textbackslash{part\{a\}} 
%         \textit{with output }\\
%         {\color{red}\textbf{Part I}\\ \textbf{a}}\\      
%     } 
% }\\

% Now, our after redefining this command we get the following 
% output:
% \framebox{
%     \parbox[c][3.0cm]{4.50cm}{
%         \addvspace{0.2cm} \centering 
%         {\color{red}\part{a} "with text capabilities" imediately following the 
%         command.}  \\
%     } 
% }\\

% \textbackslash{providecommand}\{\}\{\} which is similiar to 
% \textbackslash{newcommand}\{\}\{\} without a generated 
% error message for the existence of a command. \\ 


% \textbackslash{renewcommand}\textbackslash{\text{part[1]}}\\
% \{\textbackslash{vspace}\{.10in\} \\ 
% \textbackslash{textbf}\{\text{(\#1)}\}\} \\

% %----------------------------------------------------------------------
% \section{NFA's and DFA's}
% Ofcoase!! \LaTeX{} has an library for almost everything. The specific
% library needed to draw Finite State Machines (FSM) is the 
% \textit{tike\-automata} library. Just like we have been doing, 
% \textbackslash{usepackage}\{tikz\} must be defined in the preamble.
% Now, for the correct positioning and arrows to show direction we must
% also use the \textbackslash{usetikzlibrary}\{automata, positioning, arrows\}.\\
% Nodes are what we are going to be working with, so we want to connect
% these nodes in such a way that we get some FSM. Here is a Deterministic
% Finite Automata for some fun:
% \begin{tikzpicture}
%     \node[state, initial] (q1) {$q_1$};
%     \node[state, accepting, right of=q1] (q2) {$q_2$};
%     \node[state, right of=q2] (q3) {$q_3$};
%     \node[state, accepting, right of=q3] (q4) {$q_4$};

%     \draw  (q1) edge[loop above] node{0} (q1)
%            (q1) edge[above] node{1} (q2)
%            (q2) edge[loop above] node{1} (q2)
%            (q2) edge[bend left, above] node{0}  (q3)
%            (q3) edge[loop above] node{2} (q3)
%            (q3) edge[bend left, above] node{0}  (q4)
%            (q4) edge[bend left, below] node{0,1} (q3);
% \end{tikzpicture}
% \cite{gtikz} \\

% Here is code used:  \\
% \textbackslash{begin}\{tikzpicture\} \\
%     \textbackslash{node}[state, initial] (q1) \{\$q_1\$\}; \\
%     \textbackslash{node}[state, accepting,right of q1] (q2) \{\$q_2\$\}; \\
%     \textbackslash{node}[state, accepting,right of q1] (q2) \{\$q_2\$\}; \\
%     \textbackslash{node}[state, right of q2] (q3) \{\$q_3\$\}; \\
%     \textbackslash{node}[state, accepting,right of q3] (q4) \{\$q_4\$\}; \\
%     \textbackslash{draw}   (q1) edge[loop above] node{0} (q1) \\
%                            (q1) edge[above] node\{1\} (q2) \\
%                            (q2) edge[loop above] node\{1\} (q2) \\
%                            (q2) edge[bend left, above] node\{0\}  (q3) \\
%                            (q3) edge[loop above] node\{2\} (q3) \\
%                            (q3) edge[bend left, above] node\{0\}  (q4) \\
%                            (q4) edge[bend left, below] node\{0,1\} (q3); \\
% \textbackslash{end}\{tikzpicture\}  \\


% \section{Bibliography} ~\cite{tugorg}
% The ability to cite different sources, inspirations and influence can
% be very important if you want to be respected in your field. \\

% Ofcoarse, \LaTeX{} does have a bibliography environment 
% called "the bibliography" but after doing much research it
% is much more flexable to create a .bib file, also known as
% a \BibTex{} file that can dynamically recieve entries, after
% we identify the type of source we are referencing. 
% ~\cite{wiki_title_maker}

% All references must begin with a "@" followed by type of reference.
% There are many different types of predefined \textit{standard templates}.
% To list a few: @book\{ \}, @article\{ \}, and @misc\{ \}. \\

% These are the entries in your .bib file that you will be inserting
% in-between \textbackslash{begin}\{thebibliography\} and 
% \textbackslash{end}\{thebibliography\}. \\
% Lastly, for \LaTeX{} to be able to find your .bib file, 
% the following two lines of code 
% should be placed within you .tex file preceding \textbackslash{end}\{document\}:
% \textbackslash{bibliography}\{ \textbf{path\_to\_your\_.bib}\} and
% \textbackslash{bibliographystyle}\{ \textbf{plain} \}. 
% Now, in order to cite your source within your article this command should follow
% any instance of anothers work \textbackslash{cite}\{ref1,ref2,$\dots$,refn\}.




% %\bibliographystyle{plain}
% \bibliographystyle{IEEEtran}
% %\bibliographystyle{plainnat}
% \bibliography{LaTeX_tutorial}
% \end{document}

%---------------------- begin preamble ----------------------
% Modify the article document class, so that it 
%  produces two columns per page in 11pt
\documentclass[11pt,twocolumn]{article}
%\usepackage{times}
\usepackage[square]{natbib} 
\usepackage{enumitem, amsmath,amssymb,amsthm}
\usepackage[a4paper]{geometry}
\usepackage{url, hyperref} 
\usepackage{fullpage}
\usepackage[svgnames]{xcolor}
\usepackage{tikz}
\usetikzlibrary{automata, positioning, arrows}

%ceated during my cmps102  class.
\renewcommand\part[1]{\vspace{.10in}\textbf{(#1)}}
\newcommand\DALGORITHM{\vspace{.10in}\textbf{Description of Algorithm: }}
\newcommand\PCORRECT{\vspace{.10in}\textbf{Proof of Correctness: }}
\newcommand\TIME{\vspace{.10in}\textbf{Analysis of Time: }}
\newcommand\SPACE{\vspace{.10in}\textbf{Analysis of Space: }}

%----------------------------------------------------------------------
% 5-7pgs
%
% Document classes (describe at least three).
% How to make a title and a list of authors and their affiliations
% (this should go beyond what was discussed in class).
% Sections, subsections, sub-subsections, paragraphs, and their 
%  labeling.
% Environments, including the tabular environment (describe at 
%  least three environments).
% Mathematical formulas.
% User-defined macros (give at least three interesting examples, all 
%  different from the ones discussed in class).
% Bibliography and citation.
% At least one additional feature of LATEX that you find worth 
%  including in your tutorial (e.g.,creating figures, graphing data, 
%  making an index for a thesis or book).
%
% Your document should contain a bibliography listing a few 
%  references with information about
%----------------------------------------------------------------------

%----------------------- end preamble -----------------------
\begin{document}
\title{Basic \LaTeX{}: Features Explained} \\
\author{James M. Vrionis\\
        CMPS-185 \\
        University of California, Santa Cruz
        \date{1-26-2018} 
        %\texttt{}
       }
\maketitle 

%----------------------------------------------------------------------
~\cite{wiki_tables}

\section{Introduction} 
Pronounced either LAH-tek or LAY-tek, 
\LaTeX{} is a \textit{free} typesetting language 
developed by \textit{Leslie Lamport}  
based on \textit{Donald E. Knuth's}  
\TeX{}.\\

Markup Language Key Features: ~\cite{rjgunning_gunning_2015} 
\begin{itemize}[label={}]
   \item Typesetting journal articles.
   \item Technical reports. 
   \item Bibliographies and indexes.
   \item and many more.
\end{itemize}

\section{Install}
\subsubsection{Ubuntu Linux}
Open Terminal to install \textbf{TEX LIVE} via the \textit{apt-get}
command: \\
% Visual of where the preamble is in latex document
\framebox{
\parbox[l][1.50cm]{6.0cm}{
\addvspace{0.2cm} \centering 
\textit{sudo apt-get install texlive-full} \\
} 
}\\

\textit{NOTE: it's always good to perform a} 
\textbf{"sudo apt-get upgrade \&\& sudo apt-get update"} 
\textit{before installing anything new.}\\

now ".tex files" can be compiled using the pdflatex $<$your file$>$.tex
as long as you in the same directory as the file. \\


\begin{center}
Alternatively, you may install from the link below \\
\textittt{http://www.tug.org/texlive/}\\
\end{center}

\subsubsection{MacOS}
Open Terminal to install \MacTeX{} using \textit{Homebrew Cast}.\\

\framebox{
\parbox[l][2.50cm]{6.5cm}{
\addvspace{0.2cm} \centering 
\textit{macbk-usr\$ brew tap caskromm/cask} \\
\textit{macbk-usr\$ brew cask install mactex} \\
} 
}\\

now ".tex files" can be compiled using the pdflatex $<$your file$>$.tex
as long as you in the same directory as the file.\\

\begin{center}
Alternatively, you may install from the link below \\
\textittt{http://www.tug.org/mactex/}\\
\end{center}

\subsubsection{Windows}
To install and use a \LaTeX{} distribution on Windows 
operating system one should install \textbf{\TeX{} Live} 
via the installer.\\


\begin{center}
\textbf{Install via link below}\\
\texittt{http://www.tug.org/texlive/} 
\end{center}


\section{Compilation}

All three Operating Systems have very similar terminal
command line arguments necessary to compile you ".tex"
file $\to$ the most commonly desired ".pdf".\\
In an open terminal simply type the following:\\
'pdflatex you\_file\_name.tex' and within the current
working directory '.pdf' file will be created.\\

Linking '.bib' files take a little more work. I will
explain the steps as follows for a file named 
blah.tex. \\ 

If you have not already done so, change the current
directory of terminal to that which contains blah.tex
as well as another file created by you called blah.bib.
As long as there are not any errors within your latex
code 4 commands make this possible:\\
pdflatex blah.tex \\
bibtex blah.aux \\
pdflatex blah.tex \\
pdflatex blah.tex\\

It is important to execute the pdflatex command twice
after bibtex blah.aux has been run.
This will create your desired pdf-file with an added
references section (assuming you have entries in the
.bib file). The last section of this document will 
contain more information about bibliographies. \\


%----------------------------------------------------------------------
% At least 3 Document Class explanations
\section{Document Classes} 
The \textbf{Preamble} in a \LaTeX{} 
document is defined as the area prior to any invocation of 
begin/end \{document\} tags. This space is used to define the 
type of document class you will be using, the page layout desired,
and user defined commands.\\ 

% Visual of where the preamble is in latex document
\framebox{
\parbox[t][3.0cm]{5.50cm}{
\addvspace{0.2cm} \centering 
\framebox{
\parbox[t][1.0cm]{4.50cm}{
\addvspace{0.2cm} \centering 
\textbig{THE}  \textbig{PREAMBLE} \\
} 
}\\
\textbf{ \textbackslash{begin}\{document\}} \\ 
\textbf{ \vdots }  \\
\textbf{ \textbackslash{end}\{document\} }
} 
}\\

This is where your \textit{Document type} is to be defined. \\

\textbf{Types of Document Classes}\\
There are nine main \LaTeX{} defined textit{Document Classes}.
They are as follows: \\
{\color{blue} \textbf{article}},
{\color{blue} \textbf{IEEEtran}},
{\color{blue} \textbf{proc}},
{\color{blue} \textbf{report}}, 
{\color{blue} \textbf{book}}, 
{\color{blue} \textbf{slides}}, 
{\color{blue} \textbf{memior}}, 
{\color{blue} \textbf{letter}}, and 
{\color{blue} \textbf{beamer}}.\\

In order to use these one of these \textit{Document Classes} one must
use the following command: 
\textbackslash{documentclass\{Document-Class\}}. \\ 
where \textit{Document-class} must be replaced with one of the nine
blue labeled document classes (show above). \\
Each of these have their own unique advantages and one package may
be already optimized for your particular use, but to use any of these 
environments they must be placed with-in the \textbf{preamble} 
of your document. \\

% The \textbf{Preamble} in a \LaTeX{} 
% document is defined as the area prior to any invocation of 
% begin/end \{document\} tags. This space is used to define the 
% type of document class you will be using, the page layout desired,
% and user defined commands.\\ 

%This is where your ~\textit{Document type} is to be defined.


% The \textbf{preamble} is defined
% as the area before you have started your document. That is done 
% using begin and end document commands. \\

%\textbackslash{begin\{ document \}}.
%----------------------------------------------------------------------
% Define article: 
% a piece of writing included with others in a newspaper, magazine, or 
% other publication
\subsection{Article}
%description:
\begin{itemize}
   \item \textbf{Description:}
      \begin{itemize}[label={}]
         \item Quite possibly the most popular \LaTeX{} class used is 
         textbf{article}. 
         \item It is mainly used for short documents and 
         journal articles it can also be used for the following: 
         articles in scientific journals, presentations, short reports,
         program documentation, invitations, homework, note-taking, 
         and many more.
      \end{itemize}
   \item \textbf{Commands:}
      \begin{itemize}[label={}]
      \item \textbackslash{part\{\}}
      \item \textbackslash{chapter\{\}}
      \item \textbackslash{section\{\}}
      \item \textbackslash{paragraph\{\}} 
      \item \textbackslash{subsection\{\}}
      \end{itemize} ~\cite{latex_wikia}
%How to use:
   \item \textbf{How to use:}
      \begin{itemize}
      \item To use this particular document class within your \LaTeX{} 
       document, the \textbackslash{documentclass\{article\}} in the 
       area defined as the preamble.
      \end{itemize}
\end{itemize}
%----------------------------------------------------------------------
~\cite{gmurray_sbalemi}
\subsection{IEEEtran} 
%description:
\begin{itemize}
   \item \textbf{Description:}
      \begin{itemize}
      \item Institute of Electrical and Electronics Engineers (IEEE)
      transactions journals and conferences are what is meant by 
      by IEEEtran document class name. This document class is 
      particularly useful for (IEEE) authors and has many useful
      commands and layout required by this organization.
      \end{itemize}
   \item \textbf{Commands:}
      \begin{itemize}[label={}]
      \item \textbackslash{thanks\{\}}
      \item \textbackslash{abstract\{\}}
      \item \textbackslash{bibliography\{\}}
      \item \textbackslash{Theorem\{\}} 
      \item \textbackslash{cite\{\}}  
      \end{itemize}
%How to use:
   \item \textbf{How to use:}
      \begin{itemize}
      \item To use this particular document class within your \LaTeX{} 
       document, the \textbackslash{documentclass\{IEEEtran\}} in the 
       area defined as the preamble.
      \end{itemize}
\end{itemize}
%----------------------------------------------------------------------

%----------------------------------------------------------------------
\subsection{book}
%description:
\begin{itemize}
   \item \textbf{Description of Book:}
      \begin{itemize}
      \item The structure of this document class
      contains "chapter" which a header 
      automatically made with pages and the name
      of section on odd pages. Pages are always
      2-sided by default.  
      \end{itemize}
   \item \textbf{Commands:}
      \begin{itemize}[label={}]
      \item \textbackslash{part\{\}}
      \item \textbackslash{chapter\{\}}
      \item \textbackslash{section\{\}}
      \item \textbackslash{paragraph\{\}} ~\cite{latex_wikia}
      \item \textbackslash{subsection\{\}}  
      \end{itemize}
%How to use:
   \item \textbf{How to use:}
      \begin{itemize}
      \item To use this particular document class within your \LaTeX{} 
       document, the \textbackslash{documentclass\{book\}} in the area
       defined as the preamble.
      \end{itemize}
\end{itemize}
%----------------------------------------------------------------------


%----------------------------------------------------------------------
% Go beyond what was discussed in class
\section{Title, Listing Multiple \\ Authors and Affiliations}

Titles in \LaTeX{} can be made using \textbackslash{maketitle} or
by using the \textbf{titlepage} environment. As a little example,
here is the code I wrote to make the title of this document:\\

\flushleft{\textbackslash{title}\{Basic\textbackslash{LaTeX\{\}}: Features Explained \} 
\textbackslash \textbackslash }\\
\textbackslash{author} \{ James M. Vrionis \textbackslash \textbackslash \\
         CMPS-185, \textbackslash \textbackslash \\
         University of California, Santa Cruz \textbackslash \textbackslash \\
      \} \textbackslash \textbackslash \\
\textbackslash{maketitle}\\

My title was set up completely using the 
\textbf{maketitle} as you can see
it outputs a very professional title.


\subsection{Title}
Titles in \LaTeX{} can be made using \textbackslash{maketitle} or
by using the \textbf{titlepage} environment. As a little example,
here is the code I wrote to make the title of this document:\\

\textbackslash{title}\{Basic \textbackslash{LaTeX\{\}}: Features Explained \} 
\textbackslash \textbackslash \\
\textbackslash{author} \{ James M. Vrionis \textbackslash \textbackslash \\
         CMPS-185, \textbackslash \textbackslash \\
         University of California, Santa Cruz \textbackslash \textbackslash \\
      \} \textbackslash \textbackslash \\
\textbackslash{maketitle}\\

This code would be placed immediately after \textbackslash{begin\{document\}}.

\subsubsection{Title Page}
Another way to accomplish a \textbf{title} 
would be to make a \textbf{title page}. 
Another example should make the use of this clearer,
so first we need to define attributes within the
preamble portion of your document.\\ 

\textbackslash{title}{Basic \textbackslash{LaTeX\{\}}: Features Explained} \\
\textbackslash{author}James M. Vrionis \thanks{Inspired by CMPS-185} \\
\textbackslash{date}{January 27,  2018} \\

Next, immediately after our \textbackslash{begin}document tag 
we can put the do the following: 

% Visual of where the preamble is in latex document
\framebox{
\parbox[t][6.0cm]{5.50cm}{
\addvspace{0.3cm} \centering 
\framebox{
\parbox[t][3.50cm]{4.50cm}{
\addvspace{0.2cm} \centering 
\textbackslash{title}\{Basic \textbackslash{LaTeX\{\}}: Features Explained\} \\
\textbackslash{author}\{James M. Vrionis \textbackslash{thanks}\{Inspired by CMPS-185\}\} \\
\textbackslash{date}\{January 27,  2018\} \\
} 
}\\
\textbf{ \textbackslash{begin}\{document\}} \\ 
\textbackslash{begin}\{titlepage\}  \\
\textbackslash{maketitle}\\
\textbf{ \textbackslash{end}\{titlepage\} }
} 
}\\

All attributes of the \textbf{Title} should be defined 
within the preamble and like we have seen before 


\subsection{Multiple Authors and Affiliations}
To list multiple Authors and Affiliations the 
\textbackslash{author} command will suffice.
Let me show you another nice example of how this 
can be accomplished: \\

\textbackslash{author}\{ \\
   Author 1 \textbackslash \textbackslash \textbackslash \\
   Company or School \textbackslash \textbackslash \\
   $\cdots$ etc. $\cdots$ \textbackslash \textbackslash \\
   \textbackslash{and} \\
   Author 2 \textbackslash{thanks{$\cdots$}} \textbackslash \textbackslash \\
   $\cdots$ etc. $\cdots$ \textbackslash \textbackslash \\
   \} \\
\textbackslash{date} \textbackslash \textbackslash \\
\textbackslash{maketitle} \\

%----------------------------------------------------------------------


%----------------------------------------------------------------------
\section{Sections}
These are the \LaTeX{} commands very specific to the organization 
of document types. A section creates a numbered title above text 
its associated block of material. The numbering system is set by 
default but is easily turned off by placing an asterisk before the 
first opening curly brace. \footnote{ Using the asterisk will remove 
this section from a Table of Contents and you must re-add, 
via \textbackslash{addcontentsline} placed to the right of  
\textbackslash{section}$\asterisk$\{SoMe\_SecTioN\} \\


By default, sections are numbered and in bold font. Say we are looking
some text and we come across bolded text \textbf{5 Sections} above the 
text we are reading, which means section 5 of our text and
anything below it, whether it be a subsection, subsubsection or 
paragraph should be related to this main idea.


\framebox{
\parbox[c][7.50cm]{5.50cm}{
\begin{itemize}[label={}]
   \item \textbf{Section 1}
   \begin{itemize}[label={}]   
      \item Paragraphs
      \item Subsection 1
      \begin{itemize}[label={}]
         \item Paragraphs
         \item Sub-subsection 1 
         \begin{itemize}[label={}]
            \item Paragraphs
         \end{itemize}
         \item Sub-subsection 2 
         \begin{itemize}[label={}]
            \item Paragraphs
         \end{itemize}
      \end{itemize}
      \item Subsection 2
   \end{itemize}
   \item \textbf{Section 2}
\end{itemize}
} 
}\\

and so on...\\ 

\textbackslash{section}\{ \textcolor{blue}{Title of Section\}}} \\

\subsection{Subsections}
A \textbf{\textbackslash{\{subsection\}}} is a child of its 
parent section. If you are in section 4.0, the following 
subsections would be 4.1, 4.2,..., 4.x where x can be any 
number larger than its predecessor.  \\

If you are in section 5.0, the following subsections would be 
5.1, 5.2,...,5. {\color{blue} x} where {\color{blue} x} can be any 
number larger than its predecessor. An easy interpretation would be 
{\color{red} Section}.{\color{blue} Subsection}\\

\subsection{Sub-subsections}
A \textbf{\textbackslash{\{subsubsection\}}} is the grandchild its some 
preceding parent section. 5.1.1 would read: section 5, subsection 1 
of subsubsection 1. Possible sub-subsections are 
5.1.1, 5.1.2, ...,5.1.{\color{DarkGreen}x}, with {\color{DarkGreen}x} to any 
number larger than its predecessor.\\ 
An easy interpretation would be {\color{red} Section}.
{\color{blue} Subsection}.{\color{DarkGreen} Sub-subsection}\\

\subsection{Paragraphs} ~\cite{latex_wikia} 
When using the \textbf{\textbackslash{begin}\{paragraph\}} 
command text is broken down into lines, 
and lines are broken down into pages. To end a 
paragraph there must be one or more blank lines 
following the text inside the paragraph.\\ 

Your text should take place of the vertical dots.\\
% Visual of where the preamble is in latex document
\framebox{
    \parbox[c][2.0cm]{3.50cm}{
        \addvspace{0.2cm} \centering 
        \textbf{\textbackslash{begin}\{paragraph\}}\\ 
        \textbf{  \vdots}\\
        \textbf{\textbackslash{end}\{paragraph\}}
    } 
}\\

\subsection{Labeling} ~\cite{wiki_books_labels_cross}
Labels in \LaTeX{} are references to previous place in your 
document enabling a much better way to point back to previous 
marked objects. \\

Objects are marked with \textbackslash{label}\{ some\_name \} 
and can be retrieved using \textbackslash{ref}\{ some\_name \} 
command. \\ 

\framebox{
    \parbox[c][3.0cm]{4.50cm}{
        \addvspace{0.2cm} \centering 
        \textbackslash{\{section\}}
        \textbf{\textbackslash{label\{\textcolor{red}{labbbel}\}}}\\ 
        \textbf{ \vdots }\\
        \textbf{\textbackslash{ref\{\textcolor{red}{labbbel}\}}}
    } 
}\\

%----------------------------------------------------------------------


%----------------------------------------------------------------------
% Describe at least three environments including tabular
% Computer code, no matter what special characters it has, may be 
% listed with the verbatim environment:
\section{Environments} ~\cite{wiki_tables}
\LaTeX{} comes equip with many different environments. Three 
environments we will look at are {\color{blue}\textbf{Tabular}},
{\color{blue}\textbf{Itemize/Enumerate}}, and 
{\color{blue}\textbf{Description}}. ~\cite{wiki_tables} \\

\subsection{Tabular} 
\textbf{Tabular} is an alignment environment with vertical
and horizontal line separating capabilities. When one or both 
of these capabilities are enabled many different arrangements
become available with different column orientations, justifications,
and unique line drawing instructions. ~\cite{aj_roberts}\\
\mbox{}\\
\mbox{}\\
\textbf{Syntax:}\\  
\textbackslash{begin}\{tabular\}[pos]\{table spec\} \\
    \textbackslash{begin}\{tabular\} \{\textcolor{red}{A}\}\\
        \&  \&  \textbackslash\textbackslash \\
    \textcolor{blue}{B}    \&  \&  \textbackslash\textbackslash \\
        \&  \&  \textbackslash\textbackslash \\
    \textbackslash{end}\{tabular\}\\
    
    \begin{enumerate}[label={\Alph*}]
        \textcolor{red}{\item Here we have a place to define columns.}
        \begin{itemize}[label={}]
            \item l for left-justified 
            \item c for centered
            \item r for right-justified
        \end{itemize}
        \textcolor{blue}{\item Here we have the number of entries in the table.}
    \end{enumerate}

%---------------------------------------------------0
% Examples.. 
%---------------------------------------------------0
\framebox{
    \parbox[c][15.5cm]{9.50cm}{
        %\addvspace{0.2cm} 
        \centering
        \textbf{No line Example:} \\ 
        %\mbox{}\\

        \begin{tabular}{l c r}
        \mbox{this is}&\mbox{ table using} &\mbox{ mbox}\\
        \mbox{L-just col}&\mbox{ center col}&\mbox{ R-just col}\\
        \mbox{using }&\mbox{ tabular }&\mbox{ envir }\\
        \end{tabular}\\
        \mbox{}\\
        \mbox{}\\

        \textbf{Same example but with an added $|$: }\\
        \mbox{}\\

        \begin{tabular}{l | c r}
        \mbox{this is}&\mbox{ table using}&\mbox{ mbox} \\
        \mbox{L-just col}&\mbox{ center col}&\mbox{ R-just col}\\
        \mbox{using }&\mbox{ tabular }&\mbox{ envir } \\
        \end{tabular}\\ 
        \mbox{}\\
        \mbox{}\\

        \textbf{Added another $|$ to previous:} \\
        \mbox{}\\

        \begin{tabular}{l | c | r}
        \mbox{this is}&\mbox{ table using}&\mbox{ mbox} \\
        \mbox{L-just col}&\mbox{ center col}&\mbox{ R-just col}\\
        \mbox{using }&\mbox{ tabular }&\mbox{ envir } \\
        \end{tabular}\\ 
        \mbox{}\\
        \mbox{}\\


        \textbf{Adding a \textbackslash{hline} to previous}: \\
        \mbox{}\\

        \begin{tabular}{l | c | r}
        \hline
        \mbox{this is}&\mbox{ table using}&\mbox{ mbox} \\
        \mbox{L-just col}&\mbox{ center col}&\mbox{ R-just col}\\
        \mbox{using }&\mbox{ tabular }&\mbox{ envir } \\
        \end{tabular}\\ 
        \mbox{}\\
        \mbox{}\\

        \textbf{Added another \textbackslash{hline} to previous:} \\
        \mbox{}\\

        \begin{tabular}{l | c | r}
        \hline
        \mbox{this is}&\mbox{ table using}&\mbox{ mbox} \\ \hline
        \mbox{L-just col}&\mbox{ center col}&\mbox{ R-just col}\\
        \mbox{using }&\mbox{ tabular }&\mbox{ envir } \\
        \end{tabular}\\  
        \mbox{}\\
        \mbox{}\\
    } 
}\\
%----------------------------------------------------------------------



%----------------------------------------------------------------------
%\newpage
\subsection{Itemize and Enumerate}
\textbf{itemize} is unordered list making environment whereas
the \textbf{Enumerate} environment is numerically ordered.\\
\framebox{
    \parbox[c][3.0cm]{4.50cm}{
        \addvspace{0.2cm} \centering 
        \textbf{\textbackslash{begin}\{{itemize}\}}\\ 
        \par \textbackslash{item} "Text goes here" \\
        \textbf{ \vdots }\\
        \textbf{\textbackslash{end}\{{itemize}\}}
    } 
}\\

\framebox{
    \parbox[c][3.0cm]{4.50cm}{
        \addvspace{0.2cm} \centering 
        \textbf{\textbackslash{begin}\{{enumerate}\}}\\ 
            \par \textbackslash{item} "Text goes here" \\
        \textbf{ \vdots }\\
        \textbf{\textbackslash{end}\{{enumerate}\}}
    } 
}\\


\subsection{Description}
\textbf{Description} is an ideal list making environment for definitions.

\framebox{
    \parbox[c][3.0cm]{4.50cm}{
        \addvspace{0.2cm} \centering 
        \textbackslash{begin}\{description\}
        \textbackslash{item}[\textbf{Word}] \\
            \par Here is the definition\\
        \textbackslash{item}[\textbf{Word2}] \\
            \par Here is the definition 2 \\
        \textbackslash{end}\{description\}        
    } 
}\\

%----------------------------------------------------------------------
\section{Mathematical Formulas}

\LaTeX{} has many different ways to write mathematical expressions. 
Using either \textbf{inline} or one of the many different 
\textit{display} modes. Displays are separate from paragraphs and 
used as its own code block whereas inline expressions can be used
with-in writing environments. \\

\begin{itemize}[label={}]
    \item \textbf{Inline:}
    \begin{itemize}[label={\bullet}]
         \item  \mbox{\$ math equation inline \$}
         \item  \textbackslash{[ math equation inline }\textbackslash{]}
         \item  \textbackslash{( math equation inline }\textbackslash{)}
    \end{itemize}
    \item \textbf{Display:}
    \begin{itemize}[label={\bullet}]
         \item \textbackslash{begin}\{equation\} \textbackslash{end}\{equation\}
         \item \textbackslash{begin}\{displaymath\} \textbackslash{end}\{displaymath\}
    \end{itemize}
\end{itemize}\\
\textbf{inline vs display mode:}\\
\begin{equation}
    \mbox{\textbackslash{begin}\{equation\}}
    \sum_{i=0}^{\infty} \frac{n}{n+1}
    \mbox{  \textbackslash{end}\{equation\}}    
\end{equation}
vs.
This is inline: $\sum_{i=0}^{\infty} \frac{n}{n+1}$ \footnote{for inline place sound eq. in \$ \$} \\ 

\textbf{code}: \\
\textbackslash{sum}\_\{$i=0$\}\^\{\textbackslash{infty}\} \textbackslash{frac}\{$n$\}\{$n+1$\} \\


%----------------------------------------------------------------------



%---------------------------------------------------------------------- 
\section{User-defined macros} ~\cite{uhedu_macros}

Creating your own \textit{commands} via chaining together different
commands. The definition of which, must be in the preamble section 
of your \textbf{.tex} file. \\
Another possible use would be to redefine a command already
defined by latex. This may lead to serious problems unlike our final
safer \textit{provide command} that defines the command 
if and only if (IFF) the command doesn't exist, otherwise no 
meaning is changed. 


\subsection{New Commands}

% \newcommand\DALGORITHM{\vspace{.10in}\textbf{Description of Algorithm: }}
% \newcommand\PCORRECT{\vspace{.10in}\textbf{Proof of Correctness: }}
% \newcommand\TIME{\vspace{.10in}\textbf{Analysis of Time: }}
% \newcommand\SPACE{\vspace{.10in}\textbf{Analysis of Space: }}

\textbackslash{newcommand}\textbackslash{DALGO}\{\textbackslash{vspace}\{.10in\}
\textbackslash{textbf}\{Description of Algorithm:\}\} \\

I made this simple example while in cmps\-102 class. 
The \textbf{new command} defined here is \textbackslash{DALGO}. \\
\begin{itemize}
    \item \textbackslash{DALGO} produces the text "Description of Algorithm:"
    \item \textbackslash{newcommand} says we are creating a new command, 
    \item \textbackslash{DALGO} named "DALGO",
    \item \textbackslash{vspace}\{.10in\} .10inches 
            of vertical space between this command 
            and anything following an instance of 
            \textbackslash{DALGO},
    \item \{Description of Algorithm:\} which prints "Description of Algorithm: 
\end{itemize}

This can be really useful when using the same material over and 
over again. \LaTeX{} also gives users the ability to redefine 
existing commands via the \textbackslash{renewcommand}.
An example of how this can be accomplished is as follows: \\

Original command is \textbackslash{part\{a\}}. the output of the 
unmodified command is\\ \textbf{Part I}\\ \textbf{a}\\
Now, our after redefining this command we get the following 
output:
\part{a} "with text capabilities" immediately following the 
command.  \\

\framebox{
    \parbox[c][3.0cm]{4.50cm}{
        \addvspace{0.2cm} \centering 
        \textbf{Original Command }\\
        \textbackslash{part\{a\}} 
        \textit{with output }\\
        {\color{red}\textbf{Part I}\\ \textbf{a}}\\      
    } 
}\\

Now, our after redefining this command we get the following 
output:
\framebox{
    \parbox[c][3.0cm]{4.50cm}{
        \addvspace{0.2cm} \centering 
        {\color{red}\part{a} "with text capabilities" immediately following the 
        command.}  \\
    } 
}\\

\textbackslash{providecommand}\{\}\{\} which is similar to 
\textbackslash{newcommand}\{\}\{\} without a generated 
error message for the existence of a command. \\ 


\textbackslash{renewcommand}\textbackslash{\text{part[1]}}\\
\{\textbackslash{vspace}\{.10in\} \\ 
\textbackslash{textbf}\{\text{(\#1)}\}\} \\

%----------------------------------------------------------------------
\section{NFA's and DFA's}
Of Coase!! \LaTeX{} has a library for almost everything. The specific
library needed to draw Finite State Machines (FSM) is the 
\textit{tike\-automata} library. Just like we have been doing, 
\textbackslash{usepackage}\{tikz\} must be defined in the preamble.
Now, for the correct positioning and arrows to show direction we must
also use the \textbackslash{usetikzlibrary}\{automata, positioning, arrows\}.\\
Nodes are what we are going to be working with, so we want to connect
these nodes in such a way that we get some FSM. Here is a Deterministic
Finite Automata for some fun:
\begin{tikzpicture}
    \node[state, initial] (q1) {$q_1$};
    \node[state, accepting, right of=q1] (q2) {$q_2$};
    \node[state, right of=q2] (q3) {$q_3$};
    \node[state, accepting, right of=q3] (q4) {$q_4$};

    \draw  (q1) edge[loop above] node{0} (q1)
           (q1) edge[above] node{1} (q2)
           (q2) edge[loop above] node{1} (q2)
           (q2) edge[bend left, above] node{0}  (q3)
           (q3) edge[loop above] node{2} (q3)
           (q3) edge[bend left, above] node{0}  (q4)
           (q4) edge[bend left, below] node{0,1} (q3);
\end{tikzpicture}
\cite{gtikz} \\

Here is code used:  \\
\textbackslash{begin}\{tikzpicture\} \\
    \textbackslash{node}[state, initial] (q1) \{\$q_1\$\}; \\
    \textbackslash{node}[state, accepting,right of q1] (q2) \{\$q_2\$\}; \\
    \textbackslash{node}[state, accepting,right of q1] (q2) \{\$q_2\$\}; \\
    \textbackslash{node}[state, right of q2] (q3) \{\$q_3\$\}; \\
    \textbackslash{node}[state, accepting,right of q3] (q4) \{\$q_4\$\}; \\
    \textbackslash{draw}   (q1) edge[loop above] node{0} (q1) \\
                           (q1) edge[above] node\{1\} (q2) \\
                           (q2) edge[loop above] node\{1\} (q2) \\
                           (q2) edge[bend left, above] node\{0\}  (q3) \\
                           (q3) edge[loop above] node\{2\} (q3) \\
                           (q3) edge[bend left, above] node\{0\}  (q4) \\
                           (q4) edge[bend left, below] node\{0,1\} (q3); \\
\textbackslash{end}\{tikzpicture\}  \\


\section{Bibliography} ~\cite{tugorg}
The ability to cite different sources, inspirations and influence can
be very important if you want to be respected in your field. \\

Of coarse, \LaTeX{} does have a bibliography environment 
called "the bibliography" but after doing much research it
is much more flexible to create a .bib file, also known as
a \BibTex{} file that can dynamically receive entries, after
we identify the type of source we are referencing. 
~\cite{wiki_title_maker}

All references must begin with a "@" followed by type of reference.
There are many different types of predefined \textit{standard templates}.
To list a few: @book\{ \}, @article\{ \}, and @misc\{ \}. \\

These are the entries in your .bib file that you will be inserting
in-between \textbackslash{begin}\{thebibliography\} and 
\textbackslash{end}\{thebibliography\}. \\
Lastly, for \LaTeX{} to be able to find your .bib file, 
the following two lines of code 
should be placed within you .tex file preceding \textbackslash{end}\{document\}:
\textbackslash{bibliography}\{ \textbf{path\_to\_your\_.bib}\} and
\textbackslash{bibliographystyle}\{ \textbf{plain} \}. 
Now, in order to cite your source within your article this command should follow
any instance of another's work \textbackslash{cite}\{ref1,ref2,$\dots$,refn\}.




%\bibliographystyle{plain}
\bibliographystyle{IEEEtran}
%\bibliographystyle{plainnat}
\bibliography{LaTeX_tutorial}
\end{document}



